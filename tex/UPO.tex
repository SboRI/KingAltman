\documentclass[12pt]{article}
\usepackage{amsmath}
\usepackage{graphicx}
\usepackage{hyperref}
\usepackage[latin1]{inputenc}
\usepackage{breqn}

\newcommand{\kma}{K_{mA}}
\newcommand{\kmaa}{K_{mA2}}
\newcommand{\kmb}{K_{mB}}
\newcommand{\kib}{K_{iB}}
\newcommand{\A}{[A]}
\newcommand{\B}{[B]}






\begin{document}

\section{Derivation of UPO mechanism using King-Altman algorithm}
A rate equation for the mechanism depicted in Figure \ref{Upo-cycle} was derived using the algorithm as presented by King and Altman REPLACE REF. All possible pathways for a completely reversible reaction mechanism werde derived under fulln steady state assumption by employing Wang algebra as suggested by LAM/ PRIEST reference

\begin{figure}[h]
\centering
\includegraphics[width=1\textwidth]{"PingPong Comp Inhib Katalase cycle"}
\caption{Mechanism for UPO catalyzed hydroxylation including inhibition by substrate B and the catalase cycle. All rate limiting steps (k2, k6, k7) were assumed to be irreversible.}
\end{figure}

simplifiCations.

=0: k_-6, k_-7, [P], [Q], k_-2


substitutions:

eq = with_zero.subs(k8, k_8/k_ib)

eq = eq.subs(k1, (k_1 + k2)/k_ma)

eq = eq.subs(k4, (k_4+k7)/k_ma2)

eq = eq.subs(k3, (k_3+k6)/k_mb)


\begin{dgroup}
\begin{dmath} v =\frac{N}{D}\end{dmath}
\begin{dsuspend}where\\\end{dsuspend}

\begin{dmath}N = E_{0} K_{iB} K_{mA2} [A] [B] k_{2} k_{6}\end{dmath}
\begin{dsuspend}and\end{dsuspend}
\begin{dmath}D = K_{iB} K_{mA} K_{mA2} [B] k_{6} + K_{iB} K_{mA} K_{mB} [A] k_{7} + K_{iB} K_{mA2} K_{mB} [A] k_{2} + K_{iB} K_{mA2} [A] [B] k_{2} + K_{iB} K_{mA2} [A] [B] k_{6} + K_{iB} K_{mB} [A]^{2} k_{2} + K_{iB} K_{mB} [A]^{2} k_{7} + K_{mA} K_{mA2} [B]^{2} k_{6} + K_{mA} K_{mB} [A] [B] k_{7}\end{dmath}
\begin{dsuspend}
    rearranging to:  \\  
\end{dsuspend}
\begin{dmath}
    N = E_0 K_{mA2} [A] [B] k_2 k_6
\end{dmath}
\begin{dsuspend}
    and
\end{dsuspend}
\begin{dmath}
    D = K_{mA} \left(1 + \frac{[B]}{K_{iB}}\right) \left(\kmaa \B k_6 + \kmb \A k_7 \right) \\ + 
    \kmaa \kmb \A k_2 + \kmaa \A \B \left(k_2 + k_6\right) + \kmb \A^2 \left(k_2 + k_7\right)
\end{dmath}
\end{dgroup}


\end{document}