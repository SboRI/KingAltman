\documentclass[12pt]{article}
\usepackage{amsmath}
\usepackage{graphicx}
\usepackage{hyperref}
\usepackage[latin1]{inputenc}
\usepackage{breqn}

\newcommand{\kma}{K_{mA}}
\newcommand{\kmaa}{K_{mA2}}
\newcommand{\kmb}{K_{mB}}
\newcommand{\kib}{K_{iB}}
\newcommand{\A}{[A]}
\newcommand{\B}{[B]}






\begin{document}
\section{Fitting of UPO with substrate inhibition}
\begin{dgroup}
\begin{dsuspend} Equation for Ping-Pong-Bi-Bi with substrate inhibition\\\end{dsuspend}
\begin{dmath}v = \frac{V_{max} \A \B}{\kmb \A + \kma \B \left(1+ \frac{\B}{\kib}\right) + \A \B}\end{dmath}
\end{dgroup}

\begin{figure}[h]
\label{PingPong}
\centering
\includegraphics[width=1\textwidth]{"Km Vmax PingPongSubInhib1"}
\includegraphics[width=1\textwidth]{"Km Vmax PingPongSubInhib2"}
\caption{Fit of [EBA]/[H$_2$O$_2$] vs v with Ping-Pong + Substrate Inhibition.}
\end{figure}


\section{Derivation of UPO catalase mechanism}
A rate equation for the mechanism depicted in Figure \ref{Upo-cycle} was derived using the algorithm as presented by King and Altman
% REPLACE REF.
All possible pathways for a completely reversible reaction mechanism werde derived under full steady state assumption % by employing Wang algebra as suggested by LAM/ PRIEST reference
%(Simplifications: $k_{-2}, k_{-6}, k_{-7}, [P], [Q] = 0$; Substitutions:  $\kib = \frac{k_{-8}}{k_8}, \kma = \frac{k_{-1} + k_2}{k_2}, \kmb = \frac{k_{-3}+k_6}{k_3},\kmaa = \frac{k_{-4} + k_7}{k_4}$
\begin{figure}[h]
\label{Upo-cycle}
\centering
\includegraphics[width=1\textwidth]{"PingPong Comp Inhib Katalase cycle"}
\caption{Mechanism for UPO catalyzed hydroxylation including inhibition by substrate B and the catalase cycle. All rate limiting steps ($k_2$, $k_6$, $k_7$) were assumed to be irreversible.}
\end{figure}

%substitutions:
%eq = with_zero.subs(k8, k_8/k_ib)
%eq = eq.subs(k1, (k_1 + k2)/k_ma)
%eq = eq.subs(k4, (k_4+k7)/k_ma2)
%eq = eq.subs(k3, (k_3+k6)/k_mb)


\begin{dgroup}
\begin{dsuspend} Assumptions/simplifications:\\\end{dsuspend}
\begin{dmath*}k_{-2}, k_{-6}, k_{-7}, [P], [Q] = 0 \end{dmath*}
\begin{dsuspend}Substitutions:\end{dsuspend}
\begin{dmath*}\kib = \frac{k_{-8}}{k_8} \end{dmath*}\begin{dmath*}
\kma = \frac{k_{-1} + k_2}{k_2} \end{dmath*}\begin{dmath*}
\kmb = \frac{k_{-3}+k_6}{k_3} \end{dmath*}\begin{dmath*}
\kmaa = \frac{k_{-4} + k_7}{k_4}\end{dmath*}
\begin{dsuspend}Enzyme velocity is given by:\end{dsuspend}
\begin{dmath} v =\frac{N}{D}\end{dmath}
\begin{dsuspend}where\\\end{dsuspend}

\begin{dmath}N = E_{0} K_{iB} K_{mA2} [A] [B] k_{2} k_{6}\end{dmath}
\begin{dsuspend}and\end{dsuspend}
\begin{dmath}D = K_{iB} K_{mA} K_{mA2} [B] k_{6} + K_{iB} K_{mA} K_{mB} [A] k_{7} + K_{iB} K_{mA2} K_{mB} [A] k_{2} + K_{iB} K_{mA2} [A] [B] k_{2} + K_{iB} K_{mA2} [A] [B] k_{6} + K_{iB} K_{mB} [A]^{2} k_{2} + K_{iB} K_{mB} [A]^{2} k_{7} + K_{mA} K_{mA2} [B]^{2} k_{6} + K_{mA} K_{mB} [A] [B] k_{7}\end{dmath}
\begin{dsuspend}
    rearranging to:  \\  
\end{dsuspend}
\begin{dmath}
    N = E_0 K_{mA2} [A] [B] k_2 k_6
\end{dmath}
\begin{dsuspend}
    and
\end{dsuspend}
\begin{dmath}
    D = K_{mA} \left(1 + \frac{[B]}{K_{iB}}\right) \left(\kmaa \B k_6 + \kmb \A k_7 \right) \\ + 
    \kmaa \kmb \A k_2 + \kmaa \A \B \left(k_2 + k_6\right) + \kmb \A^2 \left(k_2 + k_7\right)
\end{dmath}
\end{dgroup}


\end{document}